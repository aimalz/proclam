\documentclass[iop]{emulateapj}

\usepackage{tikz}
\usepackage{natbib}
\usepackage{amsmath}

\usetikzlibrary{shapes.geometric, arrows}
\usetikzlibrary{fit}

\tikzstyle{hyper} = [circle, text centered, draw=black]
\tikzstyle{param} = [circle, text centered, draw=black]
\tikzstyle{data} = [circle, text centered, draw=black, line width=2pt]
\tikzstyle{arrow} = [thick,->,>=stealth]

\newcommand{\myemail}{aimalz@nyu.edu}
\newcommand{\textul}{\underline}
\newcommand{\plasticc}{PLAsTiCC}

\newcommand{\aim}[1]{\textcolor{red}{#1}}

\begin{document}

\title{Selection of a metric for probabilistic transient classification}

\author{Alex Malz\altaffilmark{1}}
\email{aimalz@nyu.edu}

\author{PLAsTiCC team}

\altaffiltext{1}{Center for Cosmology and Particle Physics, Department of 
Physics,
  New York University, 726 Broadway, 9th floor, New York, NY 10003, USA}

\begin{abstract}

\end{abstract}

\keywords{data analysis --- methods: statistical\dots}

\maketitle

\section{Introduction}
\label{sec:introduction}

Brief literature review

\begin{itemize}
	\item Previous classification challenges
	\item LSST is different due to S/N and many consumers of classification 
results
\end{itemize}

This paper aims to answer the following questions:

\begin{itemize}
	\item Q: How do we best choose a classification metric for \plasticc?\\
	A: Due to low S/N, deterministic classes are inappropriate for LSST.  
We need to consider many consumers of these classifications.
	\item Q: How can we adapt familiar deterministic metrics, and what 
probabilistic alternatives are available?\\
	A: There are many options, but we focus on a few\dots
	\item Q: What metric do we use for the choice of a metric?\\
	A: Develop intuition by considering obvious cases to align with our 
goals -- this is about systematizing a choice we as experimenters make all the 
time without being self-critical.
\end{itemize}

\section{Motivation}
\label{sec:motivation}

How do we best choose a classification metric for \plasticc?

\begin{enumerate}
	\item Establish that prior challenges have not been probabilistic but 
that LSST will have to be
	\item Establish that LSST has consumers with diverse science goals
	\item Introduce the data: refer to model and validation papers but 
share overall dataset size, rough number of classes, relative proportions, and 
potential for ``other'' class(es)
\end{enumerate}

\section{Methods}
\label{sec:methods}

How can we adapt familiar deterministic metrics, and what probabilistic 
alternatives are available?

\begin{enumerate}
	\item Introduce contrast between deterministic and probabilistic
\end{enumerate}

\subsection{Deterministic}

\subsection{Probabilistic}

\subsubsection{Brier metric}

\subsubsection{Log loss}

\section{Results}
\label{sec:results}

What metric do we use for the choice of a metric?  

\begin{enumerate}
	\item Emphasize subjectivity and thoughtful balance for survey like 
LSST (high-stakes, lots of cooks)
	\item Introduce ``pathology'' framework to cultivate intuition.
\end{enumerate}

\subsection{Controls}

\subsubsection{Random guesser}

\subsubsection{Perfect classifier}

\subsection{Treatment}

\subsubsection{Tunnel vision}

\dots

\section{Discussion}
\label{sec:discussion}

Need visualizations!

\section{Conclusion}
\label{sec:conclusion}



\begin{acknowledgements}
\end{acknowledgements}

\bibliographystyle{apj}
\bibliography{references}

\end{document}
