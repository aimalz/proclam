\RequirePackage[switch, columnwise, running, mathlines, displaymath,
mathlines]{lineno}
\RequirePackage{docswitch}
% \flag is set by the user, through the makefile:
%    make note
%    make apj
% etc.
\setjournal{\flag}

\documentclass[\docopts]{\docclass}

% You could also define the document class directly
%\documentclass[]{emulateapj}

% Custom commands from LSST DESC, see texmf/styles/lsstdesc_macros.sty
\usepackage{lsstdesc_macros}

\usepackage{graphicx}
\graphicspath{{./}{./figures/}}
\bibliographystyle{apj}

% Add your own macros here:
\newcommand{\textul}[1]{\underline{#1}}

\newcommand{\aim}[1]{\textcolor{red}{#1}}
\newcommand{\changes}[1]{\textcolor{blue}{#1}}

% in case I can think of some fancy formatting later\dots
\newcommand{\lsst}{\textsc{LSST}}
\newcommand{\plasticc}{\textsc{PLAsTiCC}}
\newcommand{\proclam}{\texttt{proclam}}
\newcommand{\snmachine}{\texttt{snmachine}}
\newcommand{\snphotcc}{\textsc{SNPhotCC}}

% ======================================================================

\begin{document}
\linenumbers

\title{The Photometric \textsc{LSST} Astronomical Time-series Classification Challenge (\textsc{PLAsTiCC}): Selection of a performance metric for classification probabilities balancing diverse science goals}

\maketitlepre

\begin{abstract}

  Light curve classification is a high priority for the of transient and variable objects, a key step in any time-domain astronomy analysis.
  However, upcoming deep photometric surveys, including the Large Synoptic Survey Telescope (\textsc{LSST}), will produce a deluge of lower signal-to-noise data for which traditional labelings are inappropriate.
  Probabilistic classifications are more appropriate for the data but are incompatible with the traditional metrics used on deterministic classifications.
  Furthermore, large survey collaborations intend to use these classification probabilities for diverse science objectives, indicating a need for a metric that balances a variety of goals.
  We describe the process used to develop an optimal performance metric for an open classification challenge that seeks probabilistic classifications and must serve many scientific interests.
  The Photometric \textsc{LSST} Astronomical Time-series Classification Challenge (\textsc{PLAsTiCC}) is an open competition aiming to identify promising methods for obtaining classification probabilities of transient and variable objects by engaging a broader community both within and outside astronomy.
  Using mock classification probability submissions mimicing archetypes of those anticipated of \textsc{PLAsTiCC}, we compare the sensitivity of two metrics of classification probabilities under various weighting schemes, finding that they yield qualitatively consistent results.
  We choose as a metric for \textsc{PLAsTiCC} a weighted modification of the log-loss because it can be more meaningfully interpreted.
  Finally, we propose extensions of our procedure to ever more complex challenge goals and suggest some guiding principles for approaching the choice of a metric of probabilistic classifications.

\end{abstract}

% Keywords are ignored in the LSST DESC Note style:
\dockeys{}

\maketitlepost

% ----------------------------------------------------------------------
% 

\section{Introduction}
\label{sec:intro}

The Large Synoptic Survey Telescope (\lsst) has the potential to advance time-domain astronomy, with anticipated impacts on the study of transient and variable objects (TVs) within and beyond the Galaxy.
% Bright cosmological objects like Type Ia supernova are probes of cosmological distance (and thus the expansion of the universe).
% Core-collapse supernovae encode within their stunning demise the evolution properties of massive stars.
% Bright astrophysical transients like RR Lyrae give insight into the structure and evolution of stars.
% Active galactic nuclei probe the evolution of large massive galaxies.
% These are but some of the physical principles that are testable with the many different kinds of transients that will be delivered with the Large Synoptic Survey Telescope (LSST).
With its rapid scan strategy, exquisite depth, and many photometric filters, \lsst\ will deliver millions of transient detections, enabling unprecedented population-level studies of astronomically varying sources in some cases across cosmic time.

Science from the \lsst\ dataset, however, is contingent on distinguishing classes of astrophysical sources
 from one another. The gold standard for such identification in astronomy has traditionally been
based on the spectrum of the source. However, the volume of discovered objects in the LSST, as well as their potentially high redshifts implies that the prospect of spectroscopic follow-up of a large fraction of them is difficult
with expected spectroscopic resources. Thus several science cases (such as SN cosmology) will actively depend om classification of astrophysical sources based on the photometric light curve, and possibly a much smaller training sample/model based on a spectroscopic sub-sample.
% Using the wealth of data (for any of a multitude of science cases) requires distinguishing different classes of sources, according to the available data.
As such, there is an acute need for photometric lightcurve classifiers that can perform well on datasets data include a wide variety of sources, and where classification over the range of objects is desired, rather than challenges that focus on only one class.

Classifaction of noisy light curves is intrinsically probabilistic. Sometimes, classification can lead to identification of a category wth very high probabilities, while at other times several categories might seem approximately
less likely. The latter is clearly the likely scenario for light curves dominated by noise.
Here, we will refer to classification schemes which preserve these probabilities as `Probabilistic Classification`.
However, often one compresses this to a
 `Deterministic classification` which is equivalent to choosing a single category with probability 1.0 through som
rules, obviously leading to a loss of information. The value of this loss depends on how the classification results are subsequently used. For example, when using classification of astrophysical sources to determine the use of follow-up resources, one ultimately takes a binary decision on whether to follow-up a particular source or not. While the knowlege that an object has been identified to be a particular type with an overwhelming probability (eg 0.99) is likely different to a situation where an object is simply slightly likelier to be a particular class than the remaining classes of objects is important but could be appropriately reflected in a deterministic classifier. If the follow-up resources were abundant enough to warrant the optimization of follow-up of less well classified objects,  the information in the probabilities could be useful in resource allocation. Considering the different case of supernova cosmology from a photometric classified sample of SNe, such probabilistic classifications can be propagated to subsequent analyses of cosmological parameters allowing one to extract information from the (large) part of the sample where photometric classification failed to identify a particular type with overwhelming probability \cite{roberts_zbeams:_2017}. This implies that one should plan and investigate probabilistic classification for future surveys like LSST.
%Because of the low-signal-to-noise expected of \lsst, probabilistic classifications are more appropriate (Roberts+17) than the point estimates of previous challenges.
%Such posteriors are more valuable than point estimates, which we call deterministic classifications in this work, because of their versatility in application and encapsulation of observational and systematic error that may propagate through inference.

Classification in astronomy is often done based on images e.g.
galaxy classification \cite{2016A&C....16...34H}, supernova
classification \cite{2017ApJ...836...97C}, identifying bars in galaxies
\cite{2018MNRAS.477..894A}, separating Near Earth Asteroids from artifacts in images
\cite{2016PASJ...68..104M}, as well as light curves e.g. \cite{2016PASJ...68..104M,2017arXiv170906257M,2017CQGra..34f4003Z}, and even noise classification e.g. \textbf{check Abbot et al ref}\cite{2017CQGra..34f4003Z,2018PhRvD..97j1501G}.
The most well-studied classification problems are often binary, with only two classes.
The last such challenge viz. SNPhotCC was also restricted to a subset of classes (specifically, extragalactic explosive events).

The Photometric \lsst\ Astronomical Time-series Classification Challenge (\plasticc) aims to identify classification techniques that serve the broader astronomical community by engaging the broader community outside astronomy.
Unlike previous challenges, \plasticc\ differs in that the more comprehensive dataset includes models for well-understood classes, newly observed classes, and classes that have only been proposed to exist, to simulate serendipitous discovery anticipated of \lsst.
Additionally, \plasticc\ will join the ranks of a handful of past astronomy classification challenges hosted on \href{https://www.kaggle.com/competitions}{Kaggle}, a platform for predictive modelling, that hosts data analytics competitions where seasoned professionals and amateurs can compete to classify, model and predict large data sets uploaded by companies or scientific collaborations. It attracts a broad user base, and those without domain knowledge of astronomy may provide novel approaches to the problem of photometric classification..

A classification posterior can be propagated through population-level inference without a separate error propagation pipeline, and for the purposes of allocation of follow-up resources, a probability can always be reduced to a point estimate for the purposes of deterministic decisionmaking (such as for allocation of follow-up resources), but a deterministic classification cannot in general be used to reconstruct a probability density for an individual object.

\plasticc\ will thus accept classifiers producing classification posteriors.

However, probabilistic classifications are incompatible with the metrics of deterministic class assignments used in previous classification challenges \cite{kessler_supernova_2010, kessler_results_2010} and efforts to develop supernova classifiers \cite{2018ApJS..236....9N}.
In this context, a metric is simply a quantification of the performance of a classifier.
Notions of accuracy, purity, completeness, and other terms endemic in science are examples of traditional metrics appropriate to classification point estimates.
Many traditional classification metrics may be modified for evaluation on probabilistic classifications \cite{lochner_photometric_2016, moller_photometric_2016, hon_deep_2017, hon_detecting_2018, 2011arXiv1108.4696G} but only by reducing class probabilities to point estimates of class and evaluating those at discrete cutoffs that can affect the conclusions of the study.

If the data are simulated using a fully self-consistent forward model, a metric of the accuracy of classification probabilities relative to the true, underlying probabilities would be straightforward.
However, such a simulation procedure would require beginning with a fully populated probability space over all classes and all possible lightcurves, which is an insurmountable challenge.
Therefore, attention must be directed toward the no longer straightforward matter of defining the criterion for a winning classifier.
In the context of astronomy, concerns about the choice of metric for probabilistic classifications have been investigated only to a limited degree thus far \cite{2018SoPh..293...28F, 2017MNRAS.464.4463K}, with most approaches concentrating on the `standard' metrics of purity and completeness, however metric consistency over a range of classifiers and between different analyses is not always ensured \cite{2018A&C....23...15B}.

This work explores the two-fold problem of how to select a metric of a probabilistic data product that will be used in many science applications and thus lacks a single obvious figure of merit.

%\aim{I actually preferred the following as a list, although I agree that it could use some pruning.}
%\begin{itemize}
%\item    The metric must return a single scalar value.
%\item    The metric must be well-defined for non-binary classes.
%\item    The metric must balance diverse science use cases in the presence of heavily nonuniform class prevalence.
%\item    The metric must respect the information content of probabilistic classifications.
%\item    The metric must be able to evaluate deterministic classifications.
%\item    The metric must be interpretable, meaning it gives a more optimal value for "good" mock classifiers and a less optimal value for mock classifiers plagued by anticipated systematic errors; in other words, it must pass basic tests of intuition.
%\item    The metric must be reliable, giving consistent results for different instantiations of the same test case.
%\end{itemize}

In order for the metric to be useful in such a heterogenous challenge, we require that the metric must return a single, scalar value, while being well-defined for non-binary classes.

In addition, the metric should balance diverse science use cases in the presence of heavily non-uniform classes.
This is key given that the rates of astornomical transients of different types varies greatly: any classifier that requires a balance of types will under perform in the \plasticc\ (and other) competitions.

We impose the restriction that the metric must respenct the information content of the probabilistic classifiers.
Should a determininstic classifier (i.e. returning a ``1'' or a ``0''), that deterministic metric must be readily convertible into a probabilistic classification - and should preserve the relationships between classes accordingly.
Similarly, any probabilistic metric much be easily transferred to a deterministic one (given, e.g. given a threshold on the most likely classficiation choice).

The metric must pass basic tests of intuition,rewarding classifiers that serve our needs and penalizing those plagued by anticipated systematic errors.

And finally, the metric must be reliable, giving consistent results for different instantiations of the same test case.
While it is clear that different procedures will happen simultansously, the metric should be stable to these changes, and be able to rank different scenarious given any particular set of rules imposed upon the metric.

The Probabilistic Classification Metric (ProClaM) code used in this exploration of performance metrics is publicly available on GitHub.\footnote{\url{https://github.com/aimalz/proclam}}


\section{Data}
\label{sec:data}

We explore the behavior of the metrics on mock classifications with well-understood weaknesses as well as realistic mock classifications from past challenges.
Data is in the form of catalogs of posterior probability vectors $p(m \mid d)$ over $M$ classes $m$ conditioned on the observed lightcurve $d$, with each probability vector normalized to sum to unity.\footnote{Throughout the paper, ```data'' always refers to classification results, not lightcurves; no \plasticc\ lightcurves were simulated, viewed, or classified in the preparation of \textit{this} paper.}
% We introduce the convention that the $M^{\mathrm{th}}$ class is designated ``other'' to encompass never-before-seen classes.
We introduce the confusion matrix $\mathbb{C}$, an $M\times M$ dimensional table of empirical probabilities $p(m \mid m')$ traditionally calculated from deterministic classification point estimates with knowledge of the true classes $m'$.
% TODO define TP/FP/TN/FN here
Though probabilistic classifications are not truly compatible with the confusion matrix, we use it as a middle ground to develop intuition from conceptually familiar deterministic classifications.

\subsection{Mock classifications}
\label{sec:mockdata}

The test cases of this section are devised to confirm that our metric aligns with our intuitive understanding of what constitutes a good classifier, that it should not reward classifications suffering from the most concerning error properties, which we will refer to as \textit{systematics} throughout the paper.
We consider a situation with $M=13$ classes
% (nominally $12$ with one designated as ``other,'' meaning not represe) and
with a lognormal distribution of the number $N_{m}$ of members of each class.

We present eight mock classifiers whose systematics are encapsulated by their confusion matrix $\mathbb{C}$.
Each classifier derives classification posteriors $p(m \mid m')$ based on the true classes as a proxy for the information contained in the lightcurves, which we hope to recover.
The posterior probability vector for a given object is a perturbation of the row of the confusion matrix corresponding to its true class, with a perturbation factor of $\delta=0.1$, following
\begin{eqnarray}
  \label{eq:cmtoprob}
  p(m \mid m') &=& \mathbb{C}_{m'} + \delta\vec{\epsilon},
\end{eqnarray}
where the perturbation vector $\vec{\epsilon}$ has components drawn from a half-Cauchy distribution
\begin{eqnarray}
  \label{eq:cauchy}
  f(x) &=& \frac{2}{\delta\pi} \left(1+\left(\frac{x+x_{0}}{\delta}\right)^{2}\right)^{-1}
\end{eqnarray}
with $x_{0}=0$.
% \aim{[Rahul: Might be good to have a plot of the pdf when x0 takes values close to 0., something like 0.4, 0.8 and close to 1.0]}

% TODO: sequential color scheme?
% TODO: letters for panels, refer to them in text

\begin{figure*}
	\begin{center}
    \includegraphics[width=0.8\textwidth]{./fig/all_sim_cm.png}
		\caption{Confusion matrices for eight mock classifiers.
    leftmost top: uncertain classification,
    leftmost bottom: perfect classifications,
    left-center top: almost perfect classification,
    left-center bottom: unbiased classifications,
    right-center top: perfect classification (type 1 and type 2 errors) for one class and uniform for all others,
    right-center bottom: assigning all objects the same class,
    rightmost top: consistently assign one class to another,
    rightmost bottom: consistently assign another class to one class}
		\label{fig:mock_cm}
	\end{center}
\end{figure*}

Figure~\ref{fig:mock_cm} shows the confusion matrices corresponding to each systematic considered, discussed in detail below.
For each case, we address:
\begin{enumerate}
  \item What defines this error property?
  \item Under what conditions is this error property relevant?
  \item What are our expectations and desires for this error property's effect on metric behavior?
\end{enumerate}

\subsubsection{Uncertain classification}
\label{sec:uncertaindata}

An entirely uniform confusion matrix $\mathbb{U}$ (leftmost top panel of Figure~\ref{fig:mock_cm}) would correspond to uniform random guesses for deterministic classification, but the probability vectors drawn from it are more nuanced.
In accordance with Equation~\ref{eq:cmtoprob}, the probability vectors are perturbations away from a uniform distribution across all classes.
The peak values of these probability vectors will correspond to uniform random classifications, however, with $p(m' \mid d)\approx M^{-1}$.
We can consider the \textit{uncertain} classifier as an experimental control for the worst possible classifier, noting that if classifications were anticorrelated with true classes, the experimenter would simply relabel them to improve performance.

\subsubsection{Accurate classification}
\label{sec:accuratedata}

The \textit{perfect} classifier has a diagonal confusion matrix $\mathbb{I}$ (leftmost bottom panel of Figure~\ref{fig:mock_cm}), which would correspond to deterministic classifications that are always correct.
In terms of probabilistic classifications, a perfect result would be a probability vector with 1 for the true class and 0 for all other classes.
Due to our addition of the random perturbation factor $\vec{e}$, the probability vectors drawn from the perfectly diagonal confusion matrix are randomly perturbed away from perfect classifications by the small factor $\delta$, but the class with maximum probability is almost always still the true class, to the tune of $\sim0.96$ with our choice of $\delta$.
This case is also a control, in that \plasticc\ would not be necessary if we believed the perfect classifier were realistically achievable.

In addition to a perfect classifier, we test linear combinations
\begin{eqnarray}
  \label{eq:lincomb}
  \mathbb{C}^{\mathrm{acc}; s} &=& \frac{1}{M+s-1} \left(s\mathbb{I} + \mathbb{U}\right)
\end{eqnarray}
of the perfect and uncertain classifier confusion matrices where the contribution of the perfect classifier is greater that of the uncertain classifier by a factor of $s$.
Deterministic classifications drawn from such a confusion matrix would be correct $s$ times as often as they are wrong, and the incorrect guesses would be uncorrelated across classes.
The probability vectors drawn from such confusion matrices would have some variability but still mostly have their peak value at the truth.
We consider the case of the \textit{almost perfect} classifier with $s=4$ (left-center top panel of Figure~\ref{fig:mock_cm}) and the \textit{noisy} classifier with $s=2$ (left-center bottom panel of Figure~\ref{fig:mock_cm}), which correspond to $p(m' \mid d)\approx0.82$ and $p(m' \mid d)\approx0.62$ respectively.
We expect that the response to variation in $s$ may differ across metrics.
(Though $s$ would realistically be expected to vary for each class, we do not conduct a systematic investigation of this possibility at this time.)

A classifier with different accuracy for each class may be considered a systematic in its own right.
An extreme example of such a classifier could be one with perfect classification performance on one class and uncertain classification on all others; its confusion matrix would be uniform except for one row, which would take a value of unity on the diagonal and zero elsewhere.
Such a classifier would have high value to those who study the class with perfect performance and should still be used in the context of \lsst.
However favoritism is inappropriate for the overall \plasticc\ metric, which must serve the needs of those who study all the classes and for different purposes.
From the perspective of \plasticc, such a \textit{tunnel vision} classifier (right-center panel of Figure~\ref{fig:mock_cm}) is a major concern.

\subsection{Inaccurate classification}
\label{sec:inaccuratedata}

Inaccurate classification in the context of a deterministic classifier is self-explanatory.
If a classifier is \textit{systematically} inaccurate, its confusion matrix has significant probability on off-diagonal elements.
We model inaccurate probabilistic classifications of class $m'$ by using the row of the confusion matrix corresponding to class $\tilde{m}$ as the basis for the perturbed probability vector
\begin{eqnarray}
  \label{eq:subsume}
  p(m\ \mid\ m') = p(m\ \mid\ \tilde{m}).
\end{eqnarray}
Class $m'$ is said to be \textit{subsumed} by class $\tilde{m}$ for such a classifier.
The asymmetric relationship between $m'$ and $\tilde{m}$ is illustrated in the rightmost top and bottom panels of Figure~\ref{fig:mock_cm}.

It is possible that the \plasticc\ classes will be subtypes of broader classes, as identifying classifiers that can distinguish between subtypes is especially relevant when the subtypes have wholly different science applications.
For example, SN Ia and SN Ibc are challenging to distinguish, and the former often subsumes the latter, an effect that can be exacerbated by underrepresentation of SN Ibc in available training sets.
However, using SN Ibc in the traditional cosmology analysis done with SN Ia can bias estimates of the cosmological parameters, so the distinction is critical.
We would like the \plasticc\ metric to identify the strength of a classifier that successfully prevents this error.

An extreme case of inaccurate classification is to classify all objects as the most common class, a particular concern for \plasticc\ given the potential for population imbalance between classes.
For our purposes, it matters not whether the subsuming class is actually the most common or only the most common in the training set.
Such a \textit{cruise control} classifier (right-center bottom panel of Figure~\ref{fig:mock_cm}) could be especially detrimental to \plasticc's goal of identifying objects belonging to extremely rare classes, particularly if the metric weights all objects equally.
Though we defend against this systematic by employing per-class weights, described in Section~\ref{sec:weights}, we nonetheless examine its impact under other plausible weighting schemes.

% \subsubsection{Combination of systematics}
% \label{sec:combo}
%
% \begin{figure}
% 	\begin{center}
% 		% \includegraphics[width=0.45\textwidth]{./fig/Combined.png}
% 		\caption{}
% 		\label{fig:combo_cm}
% 	\end{center}
% \end{figure}

\subsection{Representative classifications}
\label{sec:realdata}
%\aim{This would be more meaningful if we were given the confusion matrices of actual submissions to \snphotcc\ and then checked whether the \plasticc\ metric would have designated a different winner.
%Also, it may be more interpretable if we instead use Ashish's confusion matrices that have more classes.}

\snphotcc\ \citep{kessler_supernova_2010} focused on separating the lightcurves of a heterogenous population into a limited number of subclasses, with a goal of identifying one particular type of object for a single scientific application, and attracted diverse classification approaches, including $\chi^{2}$ fits of the SN data to publicly available templates \citep{2002PASP..114..803N} physical models \citep{2008ApJ...681..482C}, and empirical models as well as alternatives to curve-fitting such as
%, and a linear slope to magnitudes per day was used for the non-Ia sample.
outier identification on the training set Hubble diagram, dimensionality reduction,
% TODO cite InCA?
and clustering.
% A general light curve shape (rather than one motivated by the physical differences between SNeIa and core collapse SNe) was assumed by some competitors and then a kernel density estimation was performed over the fit parameters, with various approaches employed including boosting over the feature space.
Machine learning was also employed over features such as the light-curve slopes to produce a predictive model for the training data.
For more information on these methods and their success within the \snphotcc, we refer the reader to \cite{kessler_results_2010}.
In short, \snphotcc\ attracted physically motivated template-based methods sensitive to the differences between the test data and the template set
%, which are prone to bias given non-representativity of the test data and agnostic
as well as those based on decomposition of the light curves into generic features at risk of neglecting available physical information.


After the \snphotcc\ concluded, the lightcurves became a testbed for a suite of machine learning classifiers.
We consider one such compilation of methods, as presented in \cite{lochner_photometric_2016}, whose confusion matrices are shown in Figure~\ref{fig:snphotcc_cm}.
These classification algorithms include a wavelet decomposition of the spectra to construct the features over which to classify (\citet{2011MNRAS.414.1987N}, bottom row) and template-based classification procedures (\citet{2011ApJ...738..162S}, top row), each paired with Boosted Random Forest, K-Nearest Neighbors, Naive Bayes, and Support Vector Machine machine learning algorithms (columns).
While the complexity of entries to the \snphotcc\ was greater than this subset, we use these illustrative examples as a useful comparison set over which to assess the performance of the approaches under our metric scheme.

\begin{figure*}
	\begin{center}
    \includegraphics[width=\textwidth]{./fig/all_snphotcc_cm.png}
		\caption{\snphotcc\ confusion matrices.
    Top row: five machine learning methods applied to template decompositions.
    Bottom row: the same five machine learning methods applied to wavelet features.}
		\label{fig:snphotcc_cm}
	\end{center}
\end{figure*}

We draw attention to the presence of the systematics introduced in Section~\ref{sec:mockdata}.
Note that the ``WaveletsNN'' and ``WaveletsNB'' methods both suffer from the cruise control systematic.
Nearly all the others exhibit classifications that are almost perfect for the first class, perfect for the second class, and noisy for the third.

% \subsubsection{Unknown dataset}
% \label{sec:mystery}
%
% \begin{figure*}
% 	\begin{center}
% 		\includegraphics[width=0.3\textwidth]{./fig/Unknown_MLPNeuralNet_cm.png}
% 		\includegraphics[width=0.3\textwidth]{./fig/Unknown_KNeighbors_cm.png}
% 		\includegraphics[width=0.3\textwidth]{./fig/Unknown_RandomForest_cm.png}
% 		\caption{}
% 		\label{fig:unknown_cm}
% 	\end{center}
% \end{figure*}


\section{Methods}
\label{sec:methods}

%\begin{itemize}
%\item    The metric must return a single scalar value.
%\item    The metric must be well-defined for non-binary classes.
%\item    The metric must balance diverse science use cases in the presence of heavily nonuniform class prevalence.
%\item    The metric must respect the information content of probabilistic classifications.
%\item    The metric must be able to evaluate deterministic classifications.
%\item    The metric must be interpretable, meaning it gives a more optimal value for "good" mock classifiers and a less optimal value for mock classifiers plagued by anticipated systematic errors; in other words, it must pass basic tests of intuition.
%\item    The metric must be reliable, giving consistent results for different instantiations of the same test case.
%\end{itemize}

\plasticc\ aims to motivate development of improved techniques.
In order to discriminate between techniques, there must be a performance metric, a single scalar value quantifying how appropriate a classifier is for the task at hand.
Choosing a metric therefore is logically entwined with the question the challenge aims to answer.
The goal of \snphotcc\ was to identify good classifiers of SN Ia used for a photometric cosmology analysis, which lends itself to a metric that optimizes the purity of any resultant data set.
The issue of ``early classification'' for the purpose of decisionmaking regarding allocation of precious follow-up resources to spectroscopically confirm likely candidates was not addressed by the challenge participants, though it was (and remains) key to using spectroscopic resources wisely.
\plasticc, on the other hand, differs from \snphotcc\ in that its goals are not tied to one application or even one type of transient or variable class.
As such, the choice of a metric is not so simple, which has many diverse science goals with few arguments strong enough for meaningful fractions of spectroscopic follow-up.

The initial \plasticc\ challenge presents the entirety of the lightcurve for classification.
While classification of the earliest part of the lightcurve will be valued by the science team itself, the challenge is not overtly restrictive by giving classifiers only the first few points, something that will be included explicitly in upcoming versions of the challenge, to simulate the needs of early brokers for LSST.

%he posterior probabilities of class must be accurate for use in inference, and they must be evaluated in a way that does not excessively favor any one science application.

In Section~\ref{sec:science}, we review the metrics that were used in \snphotcc.
% In Section~\ref{sec:deterministic}, we review the metrics that were used in \snphotcc\ as the most similar previous challenge.
In Section~\ref{sec:probabilistic}, we introduce metrics appropriate for probabilistic classification.
We take weighted averages of the per-object metrics with per-class weights described in Section~\ref{sec:weights}.

\subsection{Science-motivated metrics}
\label{sec:science}

% TODO: Needs work to make consistent with rest of paper!

\snphotcc\ aimed to identify deterministic classifiers of SN Ia using a metric
\begin{eqnarray}
  \label{eq:snphotccfom}
  \mathcal{FOM} &\equiv& \epsilon_{Ia} \times \tilde{P}_{Ia}
\end{eqnarray}
which is the product of the \textit{efficiency}\footnote{Efficiency is generically defined as $\epsilon = \mathrm{TP} / (\mathrm{TP} + \mathrm{FN})$.}
\begin{eqnarray}
  \label{eq:efficiency}
  \epsilon_{Ia} &=& \frac{N_{Ia}^{\mathrm{true}}}{N_{Ia}^{TOT}}
\end{eqnarray}
of SN Ia classification and \textit{pseudo-purity}\footnote{Purity, also known as the positive predictive value, is generically defined as $P = \mathrm{TP} / (\mathrm{TP} + \mathrm{FP})$.}
\begin{eqnarray}
  \label{eq:pseudopurity}
  \tilde{P}_{Ia} &=& \frac{N_{Ia}^{\mathrm{true}}}{N_{Ia}^\mathrm{true} + W_{Ia}^\mathrm{false}N_{Ia}^\mathrm{false}}
\end{eqnarray}
with an additive penalty term with weight $W_{Ia}^\mathrm{false}$ motivated by the potential cosmology-parameter biases from  CC contamination.
The pseudo-purity can be interpreted as the traditional purity factor when $W_{Ia}^\mathrm{false} = 1$.

For the \snphotcc\ the penalty was related to the size of the spectroscopic subsample as roughly $W_{Ia}^\mathrm{false} = 1 + \epsilon_{spec}^{-1} \gg 1$ but the conservative limit of $W_{Ia}^\mathrm{false} = 3$ was chosen.
% to penalize wasted spectroscopic time over rejected SNe.
For future challenges, a more balanced metric can be used to ensure correct classifications across the range of classes, without focusing or highlighting a specific class as above.

% \subsection{Deterministic metrics}
% \label{sec:deterministic}

% TODO: write about AUC

\subsection{Probabilistic metrics}
\label{sec:probabilistic}

\plasticc\ aims to identify classifiers that produce discrete posterior probability distributions $p(m \mid d)$ over $M$ classes $m$ given their photometric lightcurves $d$, not deterministic classifications.
% The first difference will be discussed in Section~\ref{sec:weights}.
% TODO: discuss ROC/AUC, etc. here as well?
We consider two metrics of classification probabilities that avoid reducing probabilities to point estimates of class.
Both our metrics make use of the indicator variable
\begin{eqnarray}
  \label{eq:indicator}
  \tau_{n, m} &\equiv& \begin{cases}
  0 & m' \neq m\\
  1 & m' = m
  \end{cases}
\end{eqnarray}
for each possibile class $m$ for the $n^{\mathrm{th}}$ lightcurve which in truth belongs to class $m'$.
Both of our metrics are strictly nonnegative and approach zero for a truly perfect classifier.

\subsubsection{Log-loss}
\label{sec:logloss}

\aim{This section needs rewriting to remove jargon and clarify the discrepancies raised by Kaisey/Ashish/Tom.}
The log-loss (which is related to the cross-entropy as defined below) is a measure of how a classification algorithm performs in the case where the prediction input is a probability value between zero and unity.

\begin{eqnarray}
  \label{eq:entropy}
  H_{n} &=& -\sum_{m=1}^{M} \tau(m \mid d_{n}) \ln[\tau(m \mid d_{n})].
\end{eqnarray}
Hence $H_n$ is a measure of the entropy of the \textit{state space} of classes based on the lightcurve data, independent of knowledge of the true class. Any deterministic classification, regardless of accuracy, minimizes the entropy, and the uncertain classifier of Section~\ref{sec:uncertaindata} can be proven to maximize the entropy \citep{murphy_machine_2012}.
% TODO: fix this citation!
In an information theory context, the cross-entropy between two distributions (or in this case classes) over the same set of light curve data, describes the average number of bits needed to identify an event (lightcurve) from the set $d_n$, assuming a classification scheme is used that is optimized for an ``unnatural'' probability distribution $p$ rather than the true distribution $\tau_{n,m}$:
\begin{eqnarray}
  \label{eq:logloss}
  L_{n} &=& -\sum_{m=1}^{M}\tau_{n, m}\ln[p(m \mid d_{n})].
\end{eqnarray}
So $L_n$ measures the \textit{disorder} of using $p(m \mid d_{n})$ in place of $\tau_{n, m}$.
A difference between $L_{n}$ and $H_{n}$ evaluated at $\tau_{n}$ would be the information lost to disorder in using $p(m \mid d_{n})$ in place of $\tau_{n, m}$, also known as the Kullback-Leibler Divergence (KLD).
See \citet{malz_approximating_2018} for a comprehensive exploration of the KLD for a continuous 1-dimensional probability space.
The log-loss has only recently established a presence in the astronomy literature \citep{hon_deep_2017, hon_deep_2018}.
Its greatest strength is that it is straightforwardly interpretable, enabling the metric itself to directly contribute to uncertainty propagation in an inference problem using the probability densities provided by the classifier.
% \aim{[Rahul: Are you saying you could rewrite a BEAMs like model in terms of the logloss metric rather than the Probabilities P(m|d)?]}
% \aim{[Alex: Not rather than, but as the performance metric of how it's doing.  When propagated through a cosmology calculation, we'd be able to say that BEAMS improves the cosmological parameters by preserving X more information than the alternative.]}

\subsubsection{Brier score}
\label{sec:brier}

The Brier score \citep{brier_verification_1950}, given as
\begin{eqnarray}
  \label{eq:brier}
B_{n} &=& \sum_{m=1}^{M}(\tau_{n, m}-p(m \mid d_{n}))^{2}.
\end{eqnarray}
is a mean square error calculated between the true class indicator vector and the estimated probability vector.
It has been used extensively in solar flare forecasting \citep{crown_validation_2012, mays_ensemble_2015, florios_forecasting_2018}, stellar variability identification \citep{richards_construction_2012, armstrong_k2_2016}, and star-galaxy separation \citep{kim_hybrid_2015}.

The Brier score is an attractive option because it both rewards classifiers assigning high probability to the true class and penalizes classifiers for assigning high probability to classes other than the true class.
We expect this difference to significantly distinguish the Brier score from the log-loss.

However, its interpretation is less obvious, as its dimensions depend on those of the probability space upon which the posterior estimates are defined (classes, in the case of \plasticc).
Furthermore, modifying it with weights requires choosing whether to weight only per-object values $B_{n}$ or also the individual terms in the metric above.
We leave to future work the thorough investigation of a nontrivial weighting scheme on the Brier metric.

%\textbf{RH: this isnt super clear, we should discuss}
%\aim{I'm not sure if there's time to actually investigate the latter, nor how forthcoming to be about that being a nontrivial concern that drove the final decision about a metric for Kaggle.}

% \subsubsection{Conditional density estimation loss}
% \label{sec:cdeloss}
%
% \aim{The CDE Loss should also be included in the paper (and sent to Kaggle, if there's still time) if I can figure out how to make it work for non-ordered domains, i.e. classes.}

\subsection{Weights}
\label{sec:weights}

% TODO: motivate weights as opposed to flat
% TODO: rename section

The underlying causes of the most troubling of the systematics of Section~\ref{sec:mockdata}, tunnel vision and cruise control, are the nonrepresentative training set and an extreme imbalance of class membership rates.
Nonrepresentativity is a common problem in astronomy, as different objects naturally occur with different frequency in the universe.
Something that complicates matters even further is that classes often look alike in different scenarios.
For example, supernovae (stars that are completely destroyed in their explosions) often look like cataclysmic variables or CVs (stars that are not destroyed and whose brightness changes in a repeated way) given only the first few lightcurve points.
Similarly tidal disruption events (TDEs) that occur when stars get sufficiently close to the central black hole of a galaxy to be destroyed can look much like supernovae that are located close to the center of the galaxy.
Over longer periods of time, variations in the output from the central source of a galaxy, the active galactic nucleus (AGN) can mimic the output of a CV.

As mentioned in the previous section, subclasses often look similar but are useful in understanding very different science cases, so a metric that rewards a nuanced classifier is desireable.

\lsst\ will suffer from nonrepresentative training sets, an imbalance of classes and will see many subclasses.
These features are therefore present in \plasticc; class prevalence will differ by orders of magnitude, and there will be classes that are not present in the training set at all.
% This can be problematic because our science needs might require accurate classifications of uncommon classes as well.
Because tunnel vision is actually strong performance, it is common for tunnel classifiers to dominate challenges.
Under a ``winner takes all'' challenge and with equal weight per lightcurve, \plasticc\ would be particularly vulnerable to a tunnel vision classifier winning despite not meeting the needs of those studying less common classes.

One option is to apply a threshold of classification of all classes in order to assign an overall winner, though it would require reducing the classification probabilities to point estimates of class.
When doing binary classification with a method that reduces probabilities to point estimates of class, each object is assigned the class of higher probability, even if the two probabilities are quite similar.
This situation is particularly likely if the object, in fact, belongs to a third class or if the two classes are subclasses of a single physical phenomenon.
%We thus anticipate it to be a more severe issue for \plasticc\ and other realistic multi-class challenges as well as any challenges with multiple subclasses.

A simple reduction to a point estimate would in general be inappropriate but possibly salvageable with a threshold mechanism.
For example, requiring a minimum difference in probability density between the maximum probability class and the next highest probability class would help avert this degeneracy.
% (e.g. a newly discovered supernova with a very small number of points may be indistinguishable from a Cataclysmic variable going through a brightening).

The alternative we systematically investigate in this paper is to use a ``flat'' average of per-class metrics
\begin{eqnarray}
  \label{eq:perclassavg}
Q_{m} &=& \frac{1}{N_{M}}\sum_{n}Q_{n}
\end{eqnarray}
resulting from first averaging the metric values $Q_{n}$ for each member $n$ (i.e. each observed transient) of the class $m$ (e.g., AGN, SN Ia, RRLyrae in the astronomical example).
For this study, we consider a general weighted average
\begin{eqnarray}
  \label{eq:weightavg}
Q_{tot} &=& \frac{1}{\sum_{m}w_{m}}\sum_{m=1}^{M}w_{m}Q_{m}
\end{eqnarray}
of the per-class metrics.
Non-flat weights may be chosen to encourage challenge participants to direct more attention to classes with less active classification literature or those that have been historically more difficult to classify due to the concerning systematics.
The weights for the \plasticc\ metric, however, must be determined before the knowledge of which systematics affect which classes exists.
Because of this, the choice of weights is an inherently human problem dictated by the value placed on the scientific merits of knowledge of each class.
This paper, on the other hand, can only quantify the impact of weights in relation to the systematics.


\section{Results}
\label{sec:results}

In the following sections, we explore the response of the log-loss and Brier score metrics to the classifiers of Section~\ref{sec:data} and as a function of the weights on affected classes.

\subsection{Mock classifier systematics}
\label{sec:mockresults}

We simulate probabilistic classifications as potential submissions to \plasticc\ by the methodology of Section~\ref{sec:mockdata} based on CPMs composed of pairs of the characteristic classifiers shown in Figure~\ref{fig:all_combined} under various weightings described below.

\begin{figure*}
	\begin{center}
		\includegraphics[width=0.99\textwidth]{./fig/multipanel_res.png}
		\caption{
		Weighted log-loss and Brier scores for baseline classifiers with combinations of systematics.
		Each point represents a classifier with a shared baseline behavior (regular polygon marker; triangle for perfect, diamond for almost perfect, square for noisy) for all but one class, which is affected by a particular systematic (asterisk markers; plus for almost perfect, cross for noisy, dot for uncertain, and Y-shape for subsumed).
		The color of the marker for the systematic effect indicates the weight on the one class affected by that systematic, while the color of the baseline behavior marker indicates the integrated weight evenly distributed over other classes with baseline behavior, where lower weights are greener and higher weights are bluer.
		From left to right, we zoom in on a particular range of scores, to highlight the scale of the effect of weighted systematics on the metrics for well-behaved methods with low Brier/log-loss values.
		The ranges of Brier score and log-loss values between the panels are in ratios of approximately 10:7:3 and 100:10:5, respectively, indicating the log-loss's higher sensitivity to the presence of systematics.
		The metrics are most sensitive to the subsuming systematic on a perfect baseline (triangle with Y-shaped marker), whereas other combinations of baseline and systematic can be grouped with a smaller dynamic range in both metrics.
		}
	\end{center}
	\label{fig:all_combined}
\end{figure*}

The systematics introduced to each baseline are those that we intuitively expect to worsen classification performance of an arbitrary classifier:
\begin{itemize}
\item the uncertain, almost perfect, noisy, and subsuming classifiers are anticipated to worsen an otherwise perfect classifier;
\item the uncertain, noisy, and subsuming classifiers are anticipated to worsen an otherwise almost perfect classifier;
\item the uncertain and subsuming classifiers are anticipated to worsen an otherwise noisy classifier.
\end{itemize}
In every case, we apply the systematic to one true class, which corresponds to transforming one row of the baseline CPM.

The introduction of weights illustrates the effect each particular systematic has on a given baseline, and more importantly, how up- (or down-) weighting the affected class changes the overall metric value for the mock classifier.
Weighting schemes are defined by a weight $0 \leq w \leq 1$ on the affected class, with the remaining baseline classes sharing equal weight $(1 - w) / (M - 1)$; we test eleven weighting schemes with $w = 0., 0.1, \dots, 1.$.
A higher weight on the systematic corresponds to a lower weight on the more desirable baseline, causing both the log-loss and Brier score to increase.
This variation in weights establishes linear relationships between the log-loss and Brier score metrics for each pair of baseline and systematic, but the slope is related to the relative sensitivity of the metrics.

Figure~\ref{fig:all_combined} confirms that for all weight on the perfect classifier, the values of both metrics vanish to zero.
It is worth noting that the log-loss has more dynamic range than the Brier score overall, and that the log-loss is acutely sensitive to the subsuming systematic on a baseline of a perfect classifier.
However, the relative scales of metric values for different baseline-plus-systematic pairs are quite large, requiring three panels, zooming in from left to right.

The left panel of Figure~\ref{fig:all_combined} shows the largest variations in metric scores, for the combination of the perfect baseline and a subsuming systematic where one class is given a probability of 1 for being in another particular class and a probability of 0 for being in its true class.
This means both metrics are acutely sensitive to the subsuming systematic on a perfect baseline, which can only be overcome by aggressive downweighting.
In fact, the log-loss value for a classifier that subsumes a class into one that is classified perfectly should be infinite if the classes unaffected by the systematic have no weight; it is only finite for us because of the limits of numerical precision.

The middle panel of Figure~\ref{fig:all_combined} illustrates a narrower range of log-loss and Brier score for the subsuming systematic on the almost perfect and noisy classifier baselines.
The subsuming systematic on any baseline besides the perfect classifier defines a new regime of high but not infinite values of the metrics.

The right panel of Figure~\ref{fig:all_combined} shows the values for all other systematics on all baselines.
Though the slope is lower than in the other panels, the dynamic range of the log-loss remains higher; in other words, the log-loss is in general more sensitive to systematics than the Brier score.

In summary, both the log-loss and Brier score are most sensitive to the subsuming systematic than any other systematic.
Tuning the weights can provide an avenue toward imposing a global metric penalty on classifiers exhibiting a systematic on one class.

\begin{table}[]
\begin{tabular}{lll}
Classifier characteristic & Brier score & Log-loss\\
\hline
Perfect & 0.0 & 0.0\\
Almost perfect & 0.042 & 0.225\\
Noisy & 0.113 & 0.408\\
Uncertain & 0.253 & 0.699\\
Subsumed from Noisy & 0.447 & 1.109\\
Subsumed from Almost & 0.641 & 1.629\\
Subsumed from Perfect & 1.0 & 18.421\footnote{The entry for the log-loss of a classifier that subsumes a class into one that is otherwise perfectly classified should be infinite but is bounded by the numerical precision of our calculations.}
\end{tabular}
\caption{\sout{The value of each metric when the weight is entirely on the class with the indicated characteristic.
Weighting changes the metric performance: the value of each metric when the weight is entirely on the class with the indicated characteristic (\changes{corresponding} to a $w=1$ case in Figure~\ref{fig:all_combined}).
The log-loss is more sensitive than \changes{the} Brier score, with larger values of the score (indicating poor classification performance), particularly for the subsuming systematic.}
\changes{Metric values computed using Equation~\ref{eq:weightavg} with unit weights for the mock data produced by mock classification schemes described in Sec.~\ref{sec:mockdata}.
While the log-loss metric has a larger dynamic range than the Brier score for poor classification, the toy classifiers would be ranked the same way by either metric.}
}
\label{tab:extents}
\end{table}

When all weight is on the class exhibiting the systematic, there is a characteristic limit for each metric's values, shown in Table~\ref{tab:extents}.
Because a subsumed class takes the conditional probability vector of the subsuming class, the metric values depend on what systematics may be affecting the subsuming class as well.
While the two metrics obviously take different values, in accordance with their slopes given in Table~\ref{tab:slopes}, they do agree on the ranking of these classifiers.
Though this agreement is not in general guaranteed, it is a desirable behavior, indicating that these metrics would lead to the same conclusion about the severity of each systematic.

\begin{table}[]
\begin{tabular}{l|llll}
	& Systematics & & &\\
Baselines & Subsumed & Uncertain & Noisy & Almost\\
\hline
Perfect & 18.421 & 2.763 & 3.601 & 5.387\\
Almost perfect & 2.343 & 2.246 & 2.556 & \\
Noisy & 2.102 & 2.085 & &
\end{tabular}
\caption{
The slopes for each baseline-plus-systematic pair in the space of log-loss versus Brier score.
A higher slope corresponds to increased sensitivity of the log-loss over the Brier score.
The contrast between log-loss and Brier score is highest on a baseline of the perfect classifier, meaning the log-loss may be more appropriate for discriminating between classifiers that are already extremely good.
}
\label{tab:slopes}
\end{table}

The relative sensitivity ratios of the log-loss to the Brier score are the slopes in the trends of Figure~\ref{fig:all_combined} and are given in Table~\ref{tab:slopes}.
The log-loss always has higher sensitivity than the Brier score (i.e. it responds more strongly to up-weighting classes affected by a systematic), particularly to the difference between the perfect classifier and any lesser classifier.
A possible implication of this behavior is that the log-loss may have an enhanced ability to distinguish between multiple high-performing classifiers that might not have meaningfully different metric values under the Brier score.

On the other hand, the log-loss can be seen as more susceptible to the tunnel vision classifier because its value improves sharply with any move toward perfection.
If the subsumed class has little weight, the metric values are quite low, moreso for the log-loss than the Brier score.
This means that under a population-proportional weighting scheme, it would not be penalized for subsuming an uncommon class if it performed well for a more common class, a situation that would not serve the needs of the astronomical community.

% \aim{Still reworking past here.}
% Consider a weighting of $\sim0.8$ for a class affected by tunnel vision, leaving $\sim0.2$ to be shared evenly among all other classes uniformly affected by the other systematics.
% Qualitatively, we would say that a classifier that is almost perfect for other classes is superior to one that is noisy, and a classifier that is noisy for other classes is still superior to one that is uniform; furthermore, the subsuming classifier is even more harshly penalized in this situation than the uncertain classifier, meaning both metrics are  consistent with our basic tests of intuition in this case.
% However, this observation also indicates that the tunnel vision systematic is difficult to penalize, and that if the affected class is given a large weight, it can easily dominate the metric.
% If all classes are of scientific importance, heavily unequal weighting can incentivize tunnel vision classification.

% We introduced weighting of per-class metrics to discourage `tunnel vision' and `cruise control' classifiers that can ignore classes other than the most common and nonetheless perform well by a metric.
% Figure~\ref{fig:popweight} shows the impact of weighting the per-class metrics by the number of objects in the class as each is affected by one of the systematics and the other classes are held at the more realistic almost perfect performance.
% The points show different classification schemes, and all points are coloured by the change in the weighting, dependent on the size of the population class being classified.
% Conversely, the cruise control classifier and, to a lesser degree the noisy classifier, always has high log-loss and Brier score values regardless of the weight on the affected class.

% \begin{figure}
% 	\begin{center}
% 		\includegraphics[width=0.5\textwidth]{./fig/all_effects_isolated.png}
% 		\caption{Population-weighted log-loss and Brier scores for classifiers with one class affected by a systematic, as a function of the population of the affected class.
% 		Each point corresponds to an almost perfect classifier that with one class instead affected by a systematic (shape), with log-loss on the $y$ axis and Brier score on the $x$ axis.
% 		The metrics are calculated with a weighting (color and size) proportional to the log of its weight in a weighted average following Equation~\ref{eq:weightavg}.
% 		}
% 	\end{center}
% 	\label{fig:popweight}
% \end{figure}

% The tunnel vision classifier has a consistently low value under the Brier score and log-loss metric (bottom left corner of the plot), only increasing its Brier score once the weighting drops (less blue).
% In this view, the Brier score appears to be more susceptible to tunnel vision than the log-loss, demonstrating a more significant decrease as the size of the affected class increases, but both metrics have concerning behavior in this regard.
% This finding suggests that weighting alone may not be sufficient to counter the influence of this effect, and indicates a need for another balancing mechanism, such as requiring a threshold metric value on all classes.
% When considering a method of converting from one of these metrics to a finall `winner' of the classification challenge, care must be taken to ensure that all approaches do reasonably well at classifying more than one object.
% This thresholding procedure is discussed in the text

% \textbf{Ashish to reaplce/add more here?}
%\aim{Preliminary results indicate weighting will be very important for preventing the tunnel vision classifier from winning. It may be necessary to a priori anticipate which classes will have to be most strongly protected from this systematic via upweighting them.}

%\begin{figure}
%	\begin{center}
%		% \includegraphics[width=0.5\textwidth]{./fig/systematics_onlyperfect.png}
%		\caption{\aim{After much iteration on how best to present these tests, a figure similar to Figure~\ref{fig:cruise} but for the tunnel vision classifier (heading) on different baseline classifications (panels) as a function of weight on the affected class (rather than number of classes) is under construction.}}
%		\label{fig:tunnel}
%	\end{center}
%\end{figure}

\subsection{Representative classifications}
\label{sec:realresults}

We apply the log-loss and Brier metrics to the classification output from \snmachine. While the classification methods described in \citet{lochner_photometric_2016} refer to the idealized subset of the \snphotcc\ data, these approaches are the state-of-the-art in classification of extragalactic transients.
We present in \sout{Table~\ref{fig:snmachineresults}}\changes{Figure~\ref{fig:snmachineresults} the rankings under the} log-loss and Brier score metrics assuming an equal weight per object.
%, for classification probabilities derived from running the algorithms of \citet{lochner_photometric_2016} on the \snphotcc\ data of Section~\ref{sec:realdata}.
\sout{Table~\ref{fig:snmachineresults} also contains the ranking of classifier performance under each metric.}

% \sout{\begin{table*}[]
% 	\begin{centering}
% \begin{tabular}{lllllll}%ll}
% Rank $R$ & $R_\mathrm{FoM}$ & FoM & %$R_\mathrm{AUC}$ & AUC &
% $R_\mathrm{LogLoss}$ & Log-loss & $R_\mathrm{Brier}$ & Brier \\
% \hline
% 1  & TBDT & 0.635  %& TBDT & 0.982
% & TBDT & 0.0907 & TBDT & 0.0486 \\
% 2  & WBDT & 0.591  %& WBDT & 0.978
% & TSVM & 0.113  & TSVM & 0.0583 \\
% 3  & TSVM & 0.514  %& TSVM & 0.969
% & TNN  & 0.125  & TNN  & 0.0650 \\
% 4  & WSVM & 0.499  %& WSVM & 0.968
% & WSVM & 0.1316 & WBDT & 0.0689 \\
% 5  & TNN  & 0.496  %& TNN  & 0.954
% & WBDT & 0.1321 & WSVM & 0.0730 \\
% 6  & WNN  & 0.480  %& WNN  & 0.946
% & TKNN & 0.146  & WNN  & 0.0750 \\
% 7  & TKNN & 0.384  %& TKNN & 0.942
% & WNN  & 0.152  & TKNN & 0.0787 \\
% 8  & TNB  & 0.340  %& WKNN & 0.894
% & WKNN & 0.228  & TNB  & 0.105  \\
% 9  & WKNN & 0.114  %& TNB  & 0.879
% & TNB  & 0.251  & WKNN & 0.132  \\
% 10 & WNB  & 0.0365 %& WNB  & 0.850
% & WNB  & 0.443  & WNB  & 0.178  \\
% \end{tabular}
% 	\caption{
% 	The values of three metrics for each of ten \snmachine\ classifiers with equal weight per object.
% 	The metrics broadly agree on the ranking of the classifiers, confirming consistency between a conventional metric of classification performance and the metrics of probabilistic classifications presented here.
% 	However, there are some differences with pairwise swapping between the log-loss and Brier rankings and some significant reordering of ranks 2 through 5 with the FoM metric relative to the probabilistic metrics.
% 	}
% 	\label{tab:snmachineresults}
% 	\end{centering}
% \end{table*}}

\begin{figure}
	\begin{center}
		\includegraphics[width=0.49\textwidth]{./fig/Tables3_option4.png}
		\caption{
		\changes{The rankings of each of ten \snmachine\ classifiers with equal weight per object under the three metrics.
		The metrics broadly agree on the ranking of the classifiers, confirming consistency between a conventional metric of classification performance and the metrics of probabilistic classifications presented here.
		However, there are some differences with pairwise swapping between the log-loss and Brier rankings and some significant reordering of ranks 2 through 5 with the FoM metric relative to the probabilistic metrics.}
		}
	\end{center}
	\label{fig:snmachineresults}
\end{figure}

We apply our metrics to the classification output from \snmachine\ applied to the \snphotcc\ dataset as an example of representative light curves and representative classifiers used in extragalactic astronomy.
We present in \sout{Table~\ref{fig:snmachineresults}}\changes{Figure~\ref{fig:snmachineresults}} the rankings of each classifier under the log-loss and Brier scores assuming an equal weight per object, as well as the original \snphotcc\ metric described in Section~\ref{sec:deterministic}.
\sout{Table~\ref{fig:snmachineresults} also contains the ranking of classifier performance under each metric.}

The Brier score, log-loss, and \snphotcc\ FoM are in agreement as to the first- and last-ranked classifiers.
This consensus indicates that both of the potential \plasticc\ metrics are roughly consistent with our intuition about what makes a good classifier, providing an anchor between accepted notions of an appropriate metric and the metrics of probabilistic classifications under consideration here.
One should be careful not to generalize, however, as the rankings under the three metrics are not identical.

We note that the FoM differs more from the Brier score and log-loss metrics than they do from one another.
This is perhaps unsurprising, given that the \snphotcc\ was specifically looking to value classification algorithms that were pure (that yielded a large number of SNIa classifications and few interlopers from the other classes), as opposed to metric that rewards good performance across classes.


\section{Discussion}
\label{sec:discussion}

The goal of this work is to identify the metric most suited to \plasticc, which seeks classification posteriors of complete light curves similar to those anticipated from \lsst, with an emphasis on classification over all types, rewarding a ``best in show'' classifier rather than focusing on any one class or scientific application.\footnote{At the conclusion of \plasticc, other metrics specific to scientific uses of one or more particular classes will be used to identify ``best in class'' classification procedures that will be useful for more targeted science cases.}
The weighted log-loss is thus the metric most suited to the current \plasticc\ release.

\changes{Classification of transient and variable objects is important for a variety of scientific objectives. This diversity of scientific goals requires different trade-offs, and must be evaluated using multiple metrics. While we leave addressing such challenges for future releases of \plasticc\, and the identification of appropriate metrics to future work, we end by enumerating some of the science goals and approaches for identifying metrics.} 
\sout{Future releases of \plasticc\ will focus on different challenges in transient and variable object classification, with metrics appropriate to identifying methodologies that best enable those goals.
We discuss approaches to identifying optimal metrics for these variations, which may be developed further in future work.}

\subsection{Early classification}
\label{sec:early}

Spectroscopic follow-up is only expected of a small fraction of \lsst's detected transients and variable objects due to limited resources for such observations.
In addition to optical spectroscopic follow-up, photometric observations in other wavelength bands (near infrared and x-ray from space; microwave and radio from the ground) will be key to building a physical understanding of the object, particularly as we enter the era of multi-messenger astronomy with the added possibility of optical gravitational wave signatures.
Prompt follow-up observations are highly informative for fitting models to the light curves of familiar source classes and to characterizing anomalous light curves that could indicate never-before-seen classes that have eluded identification due to rarity or faintness.
As such, decisions about follow-up resource allocation must be made quickly and under the constraint that resources wasted on a misclassification consume the budget remaining for future follow-up attempts.
A future version of \plasticc\ focused on early light curve classification should have a metric that accounts for these limitations and rewards classifiers that perform better even when fewer observations of the lightcurve are available.

We consider the decision of whether to initiate follow-up observations to be binary and deterministic.
However, it is possible to conceive of non-binary decisions about follow-up resources; for example, one could choose between dedicating several hours on a spectroscopic instrument following up on one likely candidate or dedicating an hour each on several less likely candidates.
Here, we will discuss a metric for an early classification challenge to be focused on deterministic classification because the conversion between classification posteriors and decisions is uncharted territory that we do not explore at this time.

Even within the scope of spectroscopic follow-up as a primary motivation for early light curve classification, the goals of model-fitting to known classes and discovery of new classes would likely not share an optimal metric.
The critical question for choosing the most appropriate metric for any specific science goal motivating follow-up observations is to maximize information.
We provide two examples of the kind of information one must maximize via early light curve classification and the qualities of a deterministic metric that might enable it.

Supernova cosmology with spectroscopically confirmed light curves benefits from true positives, which contribute to the constraining power of the analysis by including one more data point;
when the class in which one is interested is as plentiful as SN Ia and our resources limited a priori, we may not be concerned by a high rate of false negatives.
% requires making a decision balancing the improved constraining power of including another SN Ia in the analysis, thereby constraining the cosmological parameters, so only true positives contribute information, and if we had a perfect classifier and standard follow-up spectroscopy resources, there would be a maximum amount of information about the cosmological parameters that could be gained in this way.
% Each false positive uses the same resources but adds no information about the cosmological parameters, and each false negative consumes no follow-up resources and deprives the Hubble diagram of one more data point.
False positives, on the other hand, may not enter the cosmology analysis, but they consume follow-up resources, thereby depriving the endeavor of the constraining power due to a single SN Ia.

A perfect classifier would lead to a maximum amount of information about the cosmological parameters conditioned on the follow-up resource budget.
For this scientific application, the metric must be chosen to not only maximize true positives but also to minimize false positives, and their relative impacts on the cosmological constraints can be quantified in terms of the information one would have about the cosmological parameters under different balances of true and false positives.
% balance the value of the information forgone by a false positive and the value of information forgone by a false negative, and the value placed on these is effectively weighted by the value we as researchers place on follow-up resources.
% \aim{Ciite some deterministic metrics relating to TP/FP?}

\subsection{Anomaly Detection}
\label{sec:anom}
Anomaly detection also gains information only from true positives, but the cost function is different in that the potential gain of information from a true positive, since there is no information about undiscovered classes ahead of time.
An example would be the recent detection of a kilonova, flagged initially by the detection of gravitational waves from an object.

Resource availability for identifying new classes is more flexible, increasing when new predictions or promising preliminary observations attract attention, and decreasing when a discovery is confirmed and the new class is established.
In this way, a false positive does not necessarily consume a resource that could otherwise be dedicated to a true positive, and the potential information gain is sufficiently great that additional resources would likely be allocated to observe the potential object.
% A false negative, on the other hand, represents forgoing an unbounded quantity of information, so minimizing the false negative rate is as important as maximizing the true positive rate.
% For a rare event like a kilonova, a false negative represents an unbounfalse positive does not appreciably reduce the amount of remaining information available to collect, but a false negative represents a large quantity of information forgone.
% Furthermore, r
% In this case, the information forgone by a false negative is significant compared to the information forgone by a false positive.
Thus, a metric tuned to anomaly detection would aim to minimize the false negative rate and maximize the true positive rate.
% \aim{Cite some deterministic metrics relating to TP/FN?}

% \subsection{Hierarchical classes}
% \label{sec:hierarchical}
%
% \aim{TODO: We would like to at some point add some content on possible ideas for extending metrics to hierarchical classes, namely conditional extensions of log-loss and possible drawbacks of penalization that can be compensated for by weighting, as well as the challenge that could pose for interpretation.}

\subsection{Difficult light curve classification}
\label{sec:difficult}

Photometric light curve classification may be challenging for a number of reasons, including the sparsity and irregularity of observations, the possible classes and how often they occur, and the distances and brightnesses of the sources of the light curves.
These factors may represent limitations on the information content of the light curves, but appropriate classifiers may be able to overcome them to a certain degree.

Though quality cuts can eliminate the most difficult light curves from entering samples used for science applications, such a practice discards information that may be of value under an analysis methodology leveraging the larger number of light curves included in a sample without cuts.
Thus, classification methods that perform well on light curves characterized by lower signal-to-noise ratios are specially important for exploiting the full potential of upcoming surveys like \lsst.

This version of \plasticc\ implements quality cuts to homogenize difficulty to some degree, and notions of classification difficulty may depend on information that will not be available until after the challenge concludes.
While the groundwork for a metric incorporating data quality has been laid by \citet{wu_radio_2018}, we defer to future work an investigation of this possibility.


\section{Conclusion}
\label{sec:conclusion}

We have presented the investigative approach to selecting an appropriate metric of the performance of classification techniques producing class posterior probabilities in the context of \plasticc.
We conclude that the Brier score and log-loss metrics could both be appropriate metrics for \plasticc\ on the basis of their responses to the most concerning systematics anticipated of competing classifiers, however, the log-loss is more sensitive to all the systematics we tested.
Both require an additional mechanism, such as weighted averaging between classes to prevent domination by a classifier that focuses exclusively on the most prevalent class, thereby failing to meet \plasticc's diverse goals.

Even though the Brier score and log-loss metrics take values consistent with one another, they are structurally and conceptually different, with wholly different interpretations.
The Brier score is a sum of square differences between probabilities, which is not physically meaningful, though the explicit penalty term is an attractive feature.
The log-loss, on the other hand, is interpretable in terms of information, meaning the metric itself can serve as a quantification of uncertainty to be propagated through forecasting of the constraining power of \lsst\ data.
We therefore choose the weighted log-loss for the overall \plasticc\ metric, with weights to be chosen on the basis of scientific merit.

We conclude by encouraging the astronomical community to continue to pursue open challenges but to think carefully about the relationship between the goals of a challenge and the global performance metric, as we have done for \plasticc, to ensure that efforts are best directed to achieve the challenge objectives.


% ----------------------------------------------------------------------

\subsection*{Acknowledgments}

%%% Here is where you should add your specific acknowledgments, remembering that some standard thanks will be added via the \code{desc-tex/ack/*.tex} and \code{contributions.tex} files.

Author contributions are listed below. \\
A.I.~Malz: conceptualization, data curation, formal analysis, investigation, methodology, project administration, software, supervision, validation, visualization, writing - original draft \\
 % Standard papers only: author contribution statements. For examples, see http://blogs.nature.com/nautilus/2007/11/post_12.html

% This work used TBD kindly provided by Not-A-DESC Member and benefitted from comments by Another Non-DESC person.

% Standard papers only: A.B.C. acknowledges support from grant 1234 from ...
This paper has undergone internal review in the LSST Dark Energy Science Collaboration. % REQUIRED if true
The authors would like to thank Melissa Graham, Weikang Lin, and Chad Schafer for serving as the LSST-DESC publication review committee.
The authors further wish to thank Kaisey Mandel, Tom Loredo, Rick Kessler, and Lluis Galbany for helpful feedback provided in the preparation of this paper.

AIM is advised by David W. Hogg and was supported by National Science Foundation grant AST-1517237.
TA is supported in part by STFC.

RB and CS are supported by the Swedish Research Council (VR) through the Oskar Klein Centre.
Their work was further supported by the research environment grant ``Gravitational Radiation and Electromagnetic Astrophysical Transients (GREAT)'' funded by the Swedish Research council (VR) under Dnr 2016-06012.

The financial assistance of the National Research Foundation (NRF) towards this research is hereby acknowledged.
Opinions expressed and conclusions arrived at, are those of the authors and are not necessarily to be attributed to the NRF.
This work is partially supported by the European Research Council under the European Community’s Seventh Framework Programme (FP7/2007-2013)/ERC grant agreement no 306478-CosmicDawn.

Canadian co-authors acknowledge support from the Natural Sciences and Engineering Research Council of Canada.
The Dunlap Institute is funded through an endowment established by the David Dunlap family and the University of Toronto.
The authors at the University of Toronto acknowledge that the land on which the University of Toronto is built is the traditional territory of the Haudenosaunee, and most recently, the territory of the Mississaugas of the New Credit First Nation.
They are grateful to have the opportunity to work in the community, on this territory.

We acknowledge the University of Chicago Research Computing Center for support of this work.
This research used resources of the National Energy Research Scientific Computing Center (NERSC), a U.S. Department of Energy Office of Science User Facility operated under Contract No. DE-AC02-05CH11231.
This research at Rutgers University is supported by US Department of Energy award DE-SC0011636.


The DESC acknowledges ongoing support from the Institut National de Physique Nucl\'eaire et de Physique des Particules in France; the Science \& Technology Facilities Council in the United Kingdom; and the Department of Energy, the National Science Foundation, and the LSST Corporation in the United States.  DESC uses resources of the IN2P3 Computing Center (CC-IN2P3--Lyon/Villeurbanne - France) funded by the Centre National de la Recherche Scientifique; the National Energy Research Scientific Computing Center, a DOE Office of Science User Facility supported by the Office of Science of the U.S.\ Department of Energy under Contract No.\ DE-AC02-05CH11231; STFC DiRAC HPC Facilities, funded by UK BIS National E-infrastructure capital grants; and the UK particle physics grid, supported by the GridPP Collaboration.  This work was performed in part under DOE Contract DE-AC02-76SF00515.
 % also available: key standard_short

% The DESC acknowledges ongoing support from the Institut National de Physique Nucleaire et de Physique des Particules in France; the Science \& Technology Facilities Council in the United Kingdom; and the Department of Energy, the National Science Foundation, and the LSST Corporation in the United States.
%
% DESC uses resources of the IN2P3 Computing Center (CC-IN2P3--Lyon/Villeurbanne - France) funded by the Centre National de la Recherche Scientifique; the National Energy Research Scientific Computing Center, a DOE Office of Science User Facility supported by the Office of Science of the U.S. Department of Energy under Contract No. DE-AC02-05CH11231; STFC DiRAC HPC Facilities, funded by UK BIS National E-infrastructure capital grants; and the UK particle physics grid, supported by the GridPP Collaboration.
%
% This work was performed in part under DOE Contract DE-AC02-76SF00515.

% We acknowledge the use of An-External-Tool-like-NED-or-ADS.

%{\it Facilities:} \facility{LSST}

% Include both collaboration papers and external citations:
\bibliography{main,lsstdesc}

\end{document}

% ======================================================================
